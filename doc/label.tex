\documentclass[12pt]{article}
\usepackage[utf8]{inputenc}
\usepackage[czech]{babel}
\begin{document}
\section*{Documentation of label.cpp}

The file is a development sandbox for estimating direction of eye gaze.
Currently, it can only find iris-like circles around an area selected by the user.
Its user interface is designed for labeling the iris position in many images, so it may feel quite weird for other purposes.

\subsection*{Formulation}

As long as the person is looking at something quite close to the camera, both the pupils and the irises will be imaged as circles.
Furthemore, they will be dark circles on bright background.
A good fitness function can be therefore formulated as average gradient magnitude along the boundary of a given circle.

In order for the fitness function to be smooth, the circle needs to be anti-aliased.
That is achieved by assigning each pixel a weight $w \in [0, 1]$ that corresponds to its distance from the circle boundary.
Pixels exactly on the boundary will be assigned $w = 1$; most of the image will be weighted as $w = 0$ and, in fact, not considered at all.

\subsection*{Update step}

The program performs a simple gradient descent algorithm on the circle position and radius.
Each step size is limited to one pixel at most, so that no important features are skipped.

\end{document}

